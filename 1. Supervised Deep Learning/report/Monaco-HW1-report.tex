\documentclass[11pt,a4paper,twocolumn]{IEEEtran}
\usepackage[utf8]{inputenc}
\usepackage{tabularx, booktabs}
\usepackage{amsmath}
\usepackage{amsfonts}
\usepackage{caption}
\usepackage{pdfpages}
\usepackage[margin=2.5cm]{geometry}
\usepackage{listings}
\usepackage{amssymb}
\usepackage{hyperref}
\usepackage{graphicx}
\usepackage{svg}
\svgsetup{inkscapeexe="C:/Program Files/Inkscape/bin/inkscape.exe"}

\usepackage{biblatex}
\addbibresource{bib1.bib}

\newcolumntype{L}[1]{>{\raggedright\let\newline\\\arraybackslash\hspace{0pt}}m{#1}}
\newcolumntype{C}[1]{>{\centering\let\newline\\\arraybackslash\hspace{0pt}}m{#1}}
\newcolumntype{R}[1]{>{\raggedleft\let\newline\\\arraybackslash\hspace{0pt}}m{#1}}

% \sepline dopo \maketitle rende tutto più carino
\newcommand{\sepline}{\noindent\makebox[\linewidth]{\rule{\textwidth}{1.2pt}}}
\newcommand{\bsepline}{\noindent\makebox[\linewidth]{\rule{7.5cm}{1.2pt}}}
\newcommand{\esepline}{\noindent\makebox[\linewidth]{\rule{7.5cm}{0.5pt}}}

\author{Monaco Saverio - 2012264 \sepline \\Neural Networks and Deep Learning - Professor: A. Testolin}
\title{{\normalsize\textsc{Università degli studi di Padova}}\vspace{-.5cm} \\ \sepline\\ \textbf{Homework \#1
\\ Supervised Deep Learning}}

\begin{document}
	\maketitle
	\begin{abstract} As for the first homework, the main models of Supervised Deep Learning are investigated. For the regression task, it is needed to effectively approximate a noisy unknown\\ 1-dimensional function. The classification task instead consists in building a Convolutional Neural Network for the FashionMNIST Dataset.\\
	For both tasks, more advanced techniques are further explored and compared.
	\end{abstract}
			%  present the simulation results
			\section{\textbf{Regression task}}
			% Describe quickly the task of regression
			The task of Regression in a Neural Network framework consist in approximating a scalar function:
			$$f:\mathbb{R}\to\mathbb{R}$$
			through the use of a Network. Usually, the deeper the network is, the more complex patterns and behaviours of the target function it can grasp, however it might also encounter overfitting.\\
			 For the current exercise, training and testing points are generated according to a theoretical and unknown function, plus some noise:
			$$\hat{y}=f(x) + noise$$
			
			The data for the task is the following:\vspace*{-.5cm}
			\begin{figure}[h]
				\centering
				\includesvg[width=0.95\linewidth]{../imgs/regression/fulldataset}
			\end{figure}
			
			\subsection{\textbf{Methods}}
			% describe your model architectures and hyperparameters
			\subsection{\textbf{Results}}
			
	\section{\textbf{Classification task}}
		% Describe quickly the task of classification
		The objective of classification is to obtain a \textit{rule} that outputs the most probable label (belonging to discrete space $\mathcal{L}$) starting from a set of parameters $\mathcal{X}$.\\
		Solely for Classification, it is possible to define a particular metric that intuitively tells how well the model is performing:
		$$Accuracy: \frac{\text{\#samples} - \text{misclassified\_samples}}{\text{\#samples}}$$
		\subsection{\textbf{Methods}}
			% describe your model architectures and hyperparameters
		\subsection{\textbf{Results}}
			%  present the simulation results
			
	\section{\textbf{Conclusions}}
	%\printbibliography
	
	\newpage
	a
	\newpage
	\onecolumn
	\section{\textbf{Appendix}}
		
	
\end{document}